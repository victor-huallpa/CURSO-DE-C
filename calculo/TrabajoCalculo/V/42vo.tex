$\displaystyle \mathbf{\int \frac{1}{x\sqrt{4x^{2}-9}}\,dx}$

\nopagebreak
$$
\begin{aligned}
\text{Operando: } &
\text{Usamos la sustitución trigonométrica } x=\tfrac{3}{2}\sec\theta. \\[6pt]
&\text{Entonces } \sec\theta=\tfrac{2x}{3},\quad 
\sqrt{4x^{2}-9}=3\tan\theta, \\[4pt]
&dx=\tfrac{3}{2}\sec\theta\tan\theta\,d\theta. \\[8pt]
\text{Sustituyendo en la integral: } &
\int \frac{1}{x\sqrt{4x^{2}-9}}\,dx
= \int \frac{1}{\big(\tfrac{3}{2}\sec\theta\big)\cdot(3\tan\theta)}
\cdot \tfrac{3}{2}\sec\theta\tan\theta\,d\theta \\[6pt]
&\quad= \int \frac{\tfrac{3}{2}\sec\theta\tan\theta}{\tfrac{9}{2}\sec\theta\tan\theta}\,d\theta
= \int \tfrac{1}{3}\,d\theta. \\[8pt]
\text{Por tanto: } &
\int \frac{1}{x\sqrt{4x^{2}-9}}\,dx = \tfrac{1}{3}\theta + C.
\end{aligned}
$$

\nopagebreak
$$
\begin{aligned}
\text{Volviendo a } x: &\quad \theta=\arctan\!\!\left(\frac{\sqrt{4x^{2}-9}}{3}\right)
\quad\text{o bien }\theta=\operatorname{arcsec}\!\!\left(\frac{2x}{3}\right). \\[6pt]
\Rightarrow\quad &
\int \frac{1}{x\sqrt{4x^{2}-9}}\,dx
= \frac{1}{3}\arctan\!\!\left(\frac{\sqrt{4x^{2}-9}}{3}\right) + C.
\end{aligned}
$$

$$
\boxed{\displaystyle 
\int \frac{1}{x\sqrt{4x^{2}-9}}\,dx
= \frac{1}{3}\arctan\!\!\left(\frac{\sqrt{4x^{2}-9}}{3}\right) + C}
$$
