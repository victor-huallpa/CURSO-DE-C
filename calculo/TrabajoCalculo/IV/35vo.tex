$\displaystyle \mathbf{\int \cos^{6}\!\left(3x\right)\,dx}$

\nopagebreak
$$
\begin{aligned}
\text{Operando: } &
\text{Usamos identidades de reducción de potencias.} \\[6pt]
&\cos^{6}\theta = \Big(\tfrac{1+\cos 2\theta}{2}\Big)^{3}
= \tfrac{1}{8}\big(1+3\cos 2\theta +3\cos^{2}2\theta +\cos^{3}2\theta\big). \\[6pt]
\text{Reducimos potencias: } &
\cos^{2}2\theta = \tfrac{1+\cos 4\theta}{2}, \\[4pt]
&\cos^{3}2\theta = \tfrac{3}{4}\cos 2\theta + \tfrac{1}{4}\cos 6\theta. \\[6pt]
\text{Sustituyendo y simplificando: } &
\cos^{6}\theta
= \tfrac{5}{16} + \tfrac{15}{32}\cos 2\theta + \tfrac{3}{16}\cos 4\theta + \tfrac{1}{32}\cos 6\theta. \\[8pt]
\text{Ahora tomamos } \theta=3x: &
\cos^{6}(3x)
= \tfrac{5}{16} + \tfrac{15}{32}\cos 6x + \tfrac{3}{16}\cos 12x + \tfrac{1}{32}\cos 18x. \\[8pt]
\text{Integramos término a término: } &
\int \cos^{6}(3x)\,dx
= \tfrac{5x}{16}
+ \tfrac{15}{32}\cdot\tfrac{\sin 6x}{6}
+ \tfrac{3}{16}\cdot\tfrac{\sin 12x}{12}
+ \tfrac{1}{32}\cdot\tfrac{\sin 18x}{18} + C \\[6pt]
&= \tfrac{5x}{16}
+ \tfrac{5}{64}\sin 6x
+ \tfrac{1}{64}\sin 12x
+ \tfrac{1}{576}\sin 18x + C.
\end{aligned}
$$

$$
\boxed{\displaystyle 
\int \cos^{6}\!\left(3x\right)\,dx
= \frac{5x}{16}
+ \frac{5}{64}\sin\!\left(6x\right)
+ \frac{1}{64}\sin\!\left(12x\right)
+ \frac{1}{576}\sin\!\left(18x\right) + C}
$$

\nopagebreak
$$
\begin{aligned}
\text{Comprobación — forma alternativa: } &\\[6pt]
\text{Multiplicamos la expresión anterior por }576: &\\[4pt]
576\cdot\Big(\tfrac{5x}{16}\Big) &= 180x, \\[2pt]
576\cdot\Big(\tfrac{5}{64}\sin 6x\Big) &= 45\sin 6x, \\[2pt]
576\cdot\Big(\tfrac{1}{64}\sin 12x\Big) &= 9\sin 12x, \\[2pt]
576\cdot\Big(\tfrac{1}{576}\sin 18x\Big) &= \sin 18x.
\end{aligned}
$$

$$
\begin{aligned}
\text{Por tanto el numerador (multiplicando por 576) es } &\\[4pt]
&180x + 45\sin 6x + 9\sin 12x + \sin 18x.
\end{aligned}
$$

$$
\begin{aligned}
\text{Usando la identidad de ángulo triple } &\sin(3\theta)=3\sin\theta-4\sin^{3}\theta,
\ \theta=6x: \\[4pt]
\sin 18x &= 3\sin 6x - 4\sin^{3}6x.
\end{aligned}
$$

$$
\begin{aligned}
\text{Sustituyendo: } &\\[4pt]
180x + 45\sin 6x + 9\sin 12x + \sin 18x
&= 180x + 45\sin 6x + 9\sin 12x + (3\sin 6x - 4\sin^{3}6x) \\[4pt]
&= 180x + 48\sin 6x + 9\sin 12x - 4\sin^{3}6x,
\end{aligned}
$$

$$
\begin{aligned}
\text{por lo que} \quad &
\frac{180x + 45\sin 6x + 9\sin 12x + \sin 18x}{576} \\[4pt]
&= \frac{180x + 48\sin 6x + 9\sin 12x - 4\sin^{3}6x}{576}.
\end{aligned}
$$

$$
\boxed{\displaystyle
\frac{5x}{16}
+ \frac{5}{64}\sin 6x
+ \frac{1}{64}\sin 12x
+ \frac{1}{576}\sin 18x
\;=\;
\frac{9 \sin(12x) - 4 \sin^{3}(6x) + 48 \sin(6x) + 180x}{576}
}
$$

