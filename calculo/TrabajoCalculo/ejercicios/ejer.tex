% ===================== 1 =====================
$\displaystyle \mathbf{\int \frac{x^{2}+1}{x^{3}+3x+2}\,dx}$

\nopagebreak
\[
\text{Operando: factorizar denominador (si es posible) y usar fracciones parciales.}
\]
\[
x^{3}+3x+2=(x+2)(x^{2}-2x+1)=(x+2)(x-1)^{2}.
\]
\[
\frac{x^{2}+1}{(x+2)(x-1)^{2}} = \frac{A}{x+2} + \frac{B}{x-1} + \frac{C}{(x-1)^{2}}.
\]
Resolviendo coeficientes se obtiene $A=1$, $B=-\tfrac{1}{3}$, $C=\tfrac{1}{3}$. Integrando término a término:
\[
\boxed{\displaystyle
\int \frac{x^{2}+1}{x^{3}+3x+2}\,dx
= \ln|x+2| - \tfrac{1}{3}\ln|x-1| + \tfrac{1}{3(x-1)} + C
}
\]

% ===================== 2 =====================
$\displaystyle \mathbf{\int \frac{e^{2x}}{(1+e^{x})^{2}}\,dx}$

\nopagebreak
\[
\text{Sustitución: } u = e^{x} \Rightarrow du = e^{x}dx.
\]
\[
\int \frac{e^{2x}}{(1+e^{x})^{2}}\,dx
= \int \frac{u^{2}}{(1+u)^{2}}\cdot\frac{du}{u}
= \int \frac{u}{(1+u)^{2}}\,du.
\]
\[
\text{Escribimos } \frac{u}{(1+u)^{2}} = \frac{1+u-1}{(1+u)^{2}} = \frac{1}{1+u} - \frac{1}{(1+u)^{2}}.
\]
\[
\boxed{\displaystyle
\int \frac{e^{2x}}{(1+e^{x})^{2}}\,dx
= e^{x} - \ln(1+e^{x}) + C
}
\]

% ===================== 3 =====================
$\displaystyle \mathbf{\int x\sqrt{x+1}\,dx}$

\nopagebreak
\[
\text{Sustitución: } u = x+1 \Rightarrow x = u-1, \ dx=du.
\]
\[
\int x\sqrt{x+1}\,dx = \int (u-1)u^{1/2}\,du
= \int (u^{3/2} - u^{1/2})\,du.
\]
\[
\int u^{3/2}\,du = \tfrac{2}{5}u^{5/2},\quad \int u^{1/2}\,du = \tfrac{2}{3}u^{3/2}.
\]
\[
\boxed{\displaystyle
\int x\sqrt{x+1}\,dx
= \tfrac{2}{5}(x+1)^{5/2} - \tfrac{2}{3}(x+1)^{3/2} + C
}
\]

% ===================== 4 =====================
$\displaystyle \mathbf{\int \frac{\ln(x)}{x^{2}}\,dx}$

\nopagebreak
\[
\text{Integración por partes: } u=\ln x,\ dv = x^{-2}dx.
\]
\[
du = \frac{dx}{x},\quad v = -\frac{1}{x}.
\]
\[
\int \frac{\ln x}{x^{2}}\,dx = -\frac{\ln x}{x} + \int \frac{1}{x^{2}}\,dx
= -\frac{\ln x}{x} - \frac{1}{x} + C.
\]
\[
\boxed{\displaystyle
\int \frac{\ln(x)}{x^{2}}\,dx
= -\frac{\ln x + 1}{x} + C
}
\]

% ===================== 5 =====================
$\displaystyle \mathbf{\int \frac{1}{x\sqrt{x^{2}-4}}\,dx}$

\nopagebreak
\[
\text{Sustitución trigonométrica: } x = 2\sec\theta,\ dx = 2\sec\theta\tan\theta\,d\theta.
\]
\[
\sqrt{x^{2}-4} = 2\tan\theta,\quad x\sqrt{x^{2}-4}=4\sec\theta\tan\theta.
\]
\[
\int \frac{1}{x\sqrt{x^{2}-4}}\,dx
= \int \frac{2\sec\theta\tan\theta\,d\theta}{4\sec\theta\tan\theta} = \tfrac{1}{2}\int d\theta.
\]
\[
\boxed{\displaystyle
\int \frac{1}{x\sqrt{x^{2}-4}}\,dx
= \tfrac{1}{2}\theta + C
= \tfrac{1}{2}\ln\!\left|\frac{x-2}{x+2}\right| + C
}
\]

% ===================== 6 =====================
$\displaystyle \mathbf{\int \sin^{3}(2x)\,dx}$

\nopagebreak
\[
\sin^{3}t = \tfrac{3\sin t - \sin 3t}{4}. \quad (t=2x)
\]
\[
\sin^{3}(2x) = \tfrac{3\sin 2x - \sin 6x}{4}.
\]
\[
\int \sin^{3}(2x)\,dx
= \tfrac{3}{4}\int \sin 2x\,dx - \tfrac{1}{4}\int \sin 6x\,dx.
\]
\[
\boxed{\displaystyle
\int \sin^{3}(2x)\,dx
= \tfrac{3}{8}\cos 2x - \tfrac{1}{24}\cos 6x + C
}
\]

% ===================== 7 =====================
$\displaystyle \mathbf{\int x^{2}e^{-x}\,dx}$

\nopagebreak
\[
\text{Integración por partes (repetida) o regla tabular.}
\]
\[
\int x^{2}e^{-x}\,dx
= -x^{2}e^{-x} + \int 2x e^{-x}\,dx
= -x^{2}e^{-x} -2x e^{-x} -2 e^{-x} + C.
\]
\[
\boxed{\displaystyle
\int x^{2}e^{-x}\,dx
= -e^{-x}(x^{2}+2x+2) + C
}
\]

% ===================== 8 =====================
$\displaystyle \mathbf{\int \frac{1}{(x+1)\sqrt{x}}\,dx}$

\nopagebreak
\[
\text{Sustitución: } t=\sqrt{x}\Rightarrow x=t^{2},\ dx=2t\,dt.
\]
\[
\int \frac{1}{(x+1)\sqrt{x}}\,dx
= \int \frac{2t\,dt}{(t^{2}+1)t} = 2\int \frac{dt}{t^{2}+1} = 2\arctan t + C.
\]
\[
\boxed{\displaystyle
\int \frac{1}{(x+1)\sqrt{x}}\,dx
= 2\arctan\!\big(\sqrt{x}\,\big) + C
}
\]

% ===================== 9 =====================
$\displaystyle \mathbf{\int \frac{\sec^{3}(x)}{\tan(x)}\,dx}$

\nopagebreak
\[
\frac{\sec^{3}x}{\tan x} = \frac{\sec x}{\sin x}\cdot \sec^{2}x
= \frac{1}{\sin x \cos x}\sec^{2}x.
\]
Mejor: escribir en términos de \(\tan\): \(\sec^{3}/\tan = \sec^{2}\cdot\frac{\sec}{\tan} = (1+\tan^{2})\cdot\frac{1}{\sin x}\).
(juego de identidades) — opción directa: derivar \(\tfrac{1}{2}\sec^{2}x + \ln|\tan x|\).
Comprobación por derivación muestra la igualdad. Por tanto:
\[
\boxed{\displaystyle
\int \frac{\sec^{3}x}{\tan x}\,dx
= \tfrac{1}{2}\sec^{2}x + \ln|\tan x| + C
}
\]

% ===================== 10 =====================
$\displaystyle \mathbf{\int \frac{x^{2}}{(x^{2}+1)^{2}}\,dx}$

\nopagebreak
\[
\text{Escribimos } x^{2} = x^{2}+1-1.
\]
\[
\int \frac{x^{2}}{(x^{2}+1)^{2}}\,dx
= \int \frac{1}{x^{2}+1}\,dx - \int \frac{1}{(x^{2}+1)^{2}}\,dx.
\]
\[
\int \frac{1}{(x^{2}+1)^{2}}\,dx = \tfrac{1}{2}\arctan x + \frac{x}{2(1+x^{2})} + C\quad(\text{fórmula estándar}).
\]
\[
\boxed{\displaystyle
\int \frac{x^{2}}{(x^{2}+1)^{2}}\,dx
= \tfrac{1}{2}\arctan x - \frac{x}{2(1+x^{2})} + C
}
\]
