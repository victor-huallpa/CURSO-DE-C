$\displaystyle \mathbf{\int \frac{1}{x^{5}+x^{4}-2x^{3}-2x^{2}+x+1}\,dx}$

\nopagebreak
\text{Operando: Factorizamos el denominador por agrupación y simplificamos:}

$$
\begin{aligned}
x^{5}+x^{4}-2x^{3}-2x^{2}+x+1 
&= (x^{5}+x^{4}) + (-2x^{3}-2x^{2}) + (x+1) \\
&= x^{4}(x+1) - 2x^{2}(x+1) + 1(x+1) \\
&= (x+1)(x^{4}-2x^{2}+1) \\
&= (x+1)\big((x^{2}-1)^{2}\big) \\
&= (x+1)^{3}(x-1)^{2}.
\end{aligned}
$$

\nopagebreak
\text{Descomposición en fracciones parciales:}

$$
\frac{1}{(x+1)^{3}(x-1)^{2}} 
= \frac{A}{x+1} + \frac{B}{(x+1)^{2}} + \frac{C}{(x+1)^{3}}
+ \frac{D}{x-1} + \frac{E}{(x-1)^{2}}.
$$

\nopagebreak
\text{Resolviendo coeficientes: }

$$
A = \frac{3}{16},\quad B = -\frac{1}{4},\quad C = -\frac{1}{8},\quad 
D = -\frac{1}{8},\quad E = -\frac{3}{16}.
$$

\nopagebreak
\text{Integramos término a término:}

$$
\begin{aligned}
\int \frac{A}{x+1}\,dx &= A \ln|x+1|, \\
\int \frac{B}{(x+1)^{2}}\,dx &= -\frac{B}{x+1}, \\
\int \frac{C}{(x+1)^{3}}\,dx &= -\frac{C}{2(x+1)^{2}}, \\
\int \frac{D}{x-1}\,dx &= D \ln|x-1|, \\
\int \frac{E}{(x-1)^{2}}\,dx &= -\frac{E}{x-1}.
\end{aligned}
$$

\nopagebreak
\text{Por tanto, la integral es:}

\[
\boxed{
\begin{aligned}
\int \frac{1}{x^{5}+x^{4}-2x^{3}-2x^{2}+x+1}\,dx
&= \frac{3}{16}\ln|x+1|
- \frac{1}{4(x+1)}
- \frac{1}{8(x+1)^{2}} \\[4pt]
&\quad - \frac{1}{8(x-1)}
- \frac{3}{16}\ln|x-1|
+ C
\end{aligned}
}
\]
