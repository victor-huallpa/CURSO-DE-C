$\displaystyle \mathbf{\int \frac{(\sec^{2}x + 1)\,\sec^{2}x}{1+\tan^{3}x}\,dx}$

\nopagebreak

\[
\text{Operando: hacemos la sustitución } u=\tan x \Rightarrow du=\sec^{2}x\,dx.
\]

\[
\text{Observación: } \sec^{2}x = 1+\tan^{2}x = 1+u^{2}.
\]

\[
\text{Sustituyendo: }
\frac{(\sec^{2}x+1)\sec^{2}x}{1+\tan^{3}x}\,dx
= \frac{(1+u^{2})+1}{1+u^{3}}\,du
= \frac{u^{2}+2}{u^{3}+1}\,du.
\]

\[
\text{Descomponemos en fracciones parciales: } 
\frac{u^{2}+2}{u^{3}+1} = \frac{A}{u+1} + \frac{Bu+C}{u^{2}-u+1}.
\]

\[
\text{Resolviendo: } A=1,\ B=0,\ C=1.
\]

\[
\text{Luego }
\frac{u^{2}+2}{u^{3}+1} = \frac{1}{u+1} + \frac{1}{u^{2}-u+1}.
\]

\[
\text{Integramos término a término: }
\int \frac{1}{u+1}\,du = \ln|u+1|.
\]

\[
\text{Para } \int \frac{1}{u^{2}-u+1}\,du
\text{ completamos el cuadrado: } u^{2}-u+1=(u-\tfrac{1}{2})^{2}+\tfrac{3}{4}.
\]

\[
\text{Por tanto }
\int \frac{1}{u^{2}-u+1}\,du
= \frac{2}{\sqrt{3}}\arctan\!\!\left(\frac{2u-1}{\sqrt{3}}\right) + C.
\]

\[
\text{Sustituyendo } u=\tan x \text{ se obtiene:}
\]

\[
\boxed{\displaystyle 
\int \frac{(\sec^{2}x + 1)\,\sec^{2}x}{1+\tan^{3}x}\,dx
= \ln\!\big(1+\tan x\big)
+ \frac{2}{\sqrt{3}}\arctan\!\!\left(\frac{2\tan x - 1}{\sqrt{3}}\right) + C.}
\]
